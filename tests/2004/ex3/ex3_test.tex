\documentclass[a4paper,10pt]{article}
\setlength{\parindent}{0ex}
\setlength{\parskip}{1.5ex}
\usepackage[dvips]{graphicx}
\usepackage{epsfig}
\usepackage[a4paper, hmargin=25mm, vmargin=30mm, nohead]{geometry}

\begin{document}
\newcommand{\marks}[1]
{\begin{flushright}{\bf (#1 marks)}\end{flushright}}

{\centering \large \bf THE UNIVERSITY OF WAIKATO\\}
{\centering \large \bf Department of Computer Science\\[0.5cm]}

{\centering \large \bf COMP201Y ---  Computer Systems\\}
{\centering \large \bf Exercise 3 Test --- 10th May 2004\\[0.3cm]}
{\centering \bf Worth 10\% --- Marked out of: 30\\[0.3cm]}
{\centering \bf Time allowed: 45 Min\\[1cm]}
\hrule

\begin{enumerate}

\item Convert the number -137.6875 to 32 bit IEEE-754 floating point
format. The format for a 32 bit IEEE-754 floating point number is given
in Figure~\ref{fig:float}. Show all working.
\marks{5}

\begin{figure}[h]
  \begin{center}
    \psfig{figure=float.eps, width=.5\linewidth}
    \caption{32bit IEEE-754 floating point format}
    \label{fig:float}
  \end{center}
\end{figure}


\item You are required to write a function \texttt{printbcd}, in WRAMP assembler. 
This function should accept a 32 bit packed BCD number passed on the stack according to WRAMP
conventions and print it to the terminal.  The routine should print the digits stored
from left to right and omit leading zeroes.  A second parameter complement number passed on the stack  indicate
the sign of the number.  If the LSB of this word is 0 the number is positive and if
it is 1 the number is negative.

You may assume the existence of a routine \texttt{putchar} which accepts an ASCII character value
passed on the stack and prints it to the terminal.

You should liberally comment your code as marks can be obtained for demonstrating appropriate
process even if your WRAMP code is not correct. 
%

\marks{25}

\end{enumerate}

The following table of ASCII codes may be useful:
\begin{center}

\begin{tabular}{|c|c|c|c|c|}
\hline	
	Decimal	& Octal & Hex & Binary	& Character \\ \hline \hline
         043   &   053  &  02B &  00101011    &    +    (plus) \\ \hline 
         044   &   054  &  02C &  00101100    &    ,    (comma) \\ \hline 
         045   &   055  &  02D &  00101101    &    -    (minus or dash) \\ \hline 
         046   &   056  &  02E &  00101110    &    .    (dot) \\ \hline 
         048   &   060  &  030 &  00110000    &    0 \\ \hline 
         049   &   061  &  031 &  00110001    &    1 \\ \hline 
         050   &   062  &  032 &  00110010    &    2 \\ \hline 
         051   &   063  &  033 &  00110011    &    3 \\ \hline 
         052   &   064  &  034 &  00110100    &    4 \\ \hline 
         053   &   065  &  035 &  00110101    &    5 \\ \hline 
         054   &   066  &  036 &  00110110    &    6 \\ \hline 
         055   &   067  &  037 &  00110111    &    7 \\ \hline 
         056   &   070  &  038 &  00111000    &    8 \\ \hline 
         057   &   071  &  039 &  00111001    &    9 \\ \hline 

\end{tabular}
\end{center}



\clearpage
\newpage
\newpage
\newpage
\setcounter{page}{1} 

\vspace*{-1cm} 

{\centering
\Large 
Department of Computer Science\\
University of Waikato\\[5mm]
COMP201Y --- Computer Systems\\[5mm]
\bf Exercise 3 Test 2004\\
Answer Sheet\\[5mm]
}
If you need more space than is provided, write on the reverse of the
page and clearly indicate this.\\[5mm]
Name:\hspace*{5cm}ID Number:\\
\hrule

\begin{enumerate}

\item Convert -137.6875 to a 32 bit IEEE-754 floating point value. Show
all working.
\marks{5}
\vspace{3mm}\hrule\vspace{7mm}\hrule\vspace{7mm}\hrule\vspace{7mm}\hrule\vspace{7mm}
\hrule\vspace{7mm}\hrule\vspace{7mm}\hrule\vspace{7mm}\hrule\vspace{7mm}
\hrule\vspace{3mm}

\newpage 

\item Write the \texttt{printbcd} function to print BCD numbers in the following space.
\marks{25}
\vspace{5mm}
\begin{large}
\texttt{.global printbcd}\vspace{2mm}\\
\texttt{.text}\vspace{2mm}\\
\texttt{printbcd:}
\end{large}
\vspace{7mm}\hrule\vspace{7mm}\hrule\vspace{7mm}\hrule\vspace{7mm}\hrule
\vspace{7mm}\hrule\vspace{7mm}\hrule\vspace{7mm}\hrule\vspace{7mm}\hrule
\vspace{7mm}\hrule\vspace{7mm}\hrule\vspace{7mm}\hrule\vspace{7mm}\hrule
\vspace{7mm}\hrule\vspace{7mm}\hrule\vspace{7mm}\hrule\vspace{7mm}\hrule
\vspace{7mm}\hrule\vspace{7mm}\hrule\vspace{7mm}\hrule\vspace{7mm}\hrule
\vspace{7mm}\hrule\vspace{7mm}\hrule\vspace{7mm}\hrule\vspace{7mm}\hrule
\vspace{7mm}\hrule\vspace{7mm}\hrule\vspace{7mm}\hrule\vspace{7mm}\hrule
\vspace{7mm}\hrule\vspace{7mm}\hrule\vspace{7mm}\hrule\vspace{7mm}\hrule
\vspace{7mm}\hrule\vspace{7mm}\hrule\vspace{7mm}\hrule\vspace{7mm}\hrule
\vspace{7mm}\hrule\vspace{7mm}\hrule\vspace{7mm}\hrule\vspace{7mm}\hrule
\vspace{7mm}\hrule\vspace{7mm}\hrule\vspace{7mm}\hrule\vspace{7mm}\hrule
\vspace{7mm}\hrule\vspace{7mm}\hrule\vspace{7mm}\hrule\vspace{7mm}\hrule
\vspace{7mm}\hrule\vspace{7mm}\hrule\vspace{7mm}\hrule\vspace{7mm}\hrule
\vspace{7mm}\hrule\vspace{7mm}\hrule\vspace{7mm}\hrule\vspace{7mm}\hrule
\vspace{3mm}

If you require more space, continue overleaf and clearly indicate that you have done
so.

\end{enumerate}

\end{document}




