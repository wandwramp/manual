\documentclass[a4paper,10pt]{article}
\setlength{\parindent}{0ex}
\setlength{\parskip}{1.5ex}
\usepackage[dvips]{graphicx}
\usepackage{epsfig}
\usepackage[a4paper, hmargin=25mm, vmargin=30mm, nohead]{geometry}

\newcommand{\marks}[1]
{\begin{flushright}{\bf (#1 marks)}\end{flushright}}

\begin{document}

\vspace*{-1cm} 

{\centering \large \bf THE UNIVERSITY OF WAIKATO\\}
{\centering \large \bf Department of Computer Science\\[0.5cm]}

{\centering \large \bf COMP201Y Computer Systems 2003 \\}
{\centering \large \bf Mid Year Test - Wednesday 4th June, 2003, 7pm L2/L3\\[1cm]}
{\centering \large \bf Answer Sheet\\[5mm]}
If you need more space than is provided, write on the reverse of the
page and clearly indicate this.\\[5mm]
Name:\hspace*{5cm}ID Number:\\
\hrule

\section{Multichoice}

\begin{tabular}{p{1.5cm}|p{2cm}|}
\cline{2-2}
1. &  \\
& \\ & \\
\cline{2-2}
2. &  \\
& \\ & \\
\cline{2-2}
3. &  \\
& \\ & \\
\cline{2-2}
4. &  \\
& \\ & \\
\cline{2-2}
5. &  \\
& \\ & \\ 
\cline{2-2}
6. &  \\
& \\ & \\
\cline{2-2}
7. &  \\
& \\ & \\
\cline{2-2}
8. &  \\
& \\ & \\
\cline{2-2}
9. &  \\
& \\ & \\
\cline{2-2}
10. &  \\
& \\ & \\
\cline{2-2}
\end{tabular}

\newpage
\section{Short Answer}
\begin{enumerate}


\item~

\vspace{7mm}\hrule\vspace{7mm}\hrule\vspace{7mm}\hrule\vspace{7mm}\hrule
\vspace{7mm}\hrule\vspace{7mm}\hrule\vspace{7mm}\hrule\vspace{3mm}

\item~

\vspace{7mm}\hrule\vspace{7mm}\hrule\vspace{7mm}\hrule\vspace{7mm}\hrule
\vspace{7mm}\hrule\vspace{3mm}

\item~

\vspace{7mm}\hrule\vspace{7mm}\hrule\vspace{7mm}\hrule\vspace{7mm}\hrule
\vspace{7mm}\hrule\vspace{7mm}\hrule\vspace{7mm}\hrule\vspace{7mm}\hrule\vspace{7mm}\hrule\vspace{3mm}

\item~

\vspace{7mm}\hrule\vspace{7mm}\hrule\vspace{7mm}\hrule\vspace{7mm}\hrule
\vspace{7mm}\hrule\vspace{3mm}

\newpage
\item~

\vspace{7mm}\hrule\vspace{7mm}\hrule\vspace{7mm}\hrule\vspace{7mm}\hrule
\vspace{7mm}\hrule\vspace{7mm}\hrule\vspace{7mm}\hrule\vspace{7mm}\hrule
\vspace{7mm}\hrule\vspace{7mm}\hrule\vspace{7mm}\hrule\vspace{3mm}

\item~

\vspace{7mm}\hrule\vspace{7mm}\hrule\vspace{7mm}\hrule\vspace{7mm}\hrule
\vspace{7mm}\hrule\vspace{7mm}\hrule\vspace{7mm}\hrule\vspace{7mm}\hrule\vspace{3mm}

\item~

\vspace{7mm}\hrule\vspace{7mm}\hrule\vspace{7mm}\hrule\vspace{7mm}\hrule
\vspace{7mm}\hrule\vspace{7mm}\hrule\vspace{7mm}\hrule\vspace{7mm}\hrule
\vspace{7mm}\hrule\vspace{3mm}

\newpage
\item~

\vspace{7mm}\hrule\vspace{7mm}\hrule\vspace{7mm}\hrule\vspace{7mm}\hrule
\vspace{7mm}\hrule\vspace{7mm}\hrule\vspace{7mm}\hrule\vspace{7mm}\hrule
\vspace{7mm}\hrule\vspace{3mm}

\item~

\vspace{7mm}\hrule\vspace{7mm}\hrule\vspace{7mm}\hrule\vspace{7mm}\hrule
\vspace{7mm}\hrule\vspace{7mm}\hrule\vspace{7mm}\hrule\vspace{3mm}

\item~

\vspace{7mm}\hrule\vspace{7mm}\hrule\vspace{7mm}\hrule\vspace{7mm}\hrule
\vspace{7mm}\hrule\vspace{7mm}\hrule\vspace{7mm}\hrule\vspace{3mm}

\item~

\vspace{7mm}\hrule\vspace{7mm}\hrule\vspace{7mm}\hrule\vspace{7mm}\hrule
\vspace{7mm}\hrule\vspace{3mm}

\newpage
\item~

\vspace{7mm}\hrule\vspace{7mm}\hrule\vspace{7mm}\hrule\vspace{7mm}\hrule\vspace{7mm}\hrule
\vspace{7mm}\hrule\vspace{7mm}\hrule\vspace{7mm}\hrule\vspace{3mm}

\item~

\vspace{7mm}\hrule\vspace{7mm}\hrule\vspace{7mm}\hrule\vspace{7mm}\hrule\vspace{7mm}\hrule
\vspace{7mm}\hrule\vspace{7mm}\hrule\vspace{7mm}\hrule\vspace{3mm}

\item~

\vspace{7mm}\hrule\vspace{7mm}\hrule\vspace{7mm}\hrule\vspace{7mm}\hrule\vspace{7mm}\hrule
\vspace{7mm}\hrule\vspace{7mm}\hrule\vspace{7mm}\hrule\vspace{7mm}\hrule
\vspace{7mm}\hrule\vspace{7mm}\hrule\vspace{7mm}\hrule\vspace{3mm}

\newpage
\item

\begin{enumerate}
 \item What are the contents of the following registers each time the
breakpoint is encountered? You should fill in one line each time you
believe a breakpoint was hit. You should not need more lines than are
provided, although you may need less.

\begin{center}
\begin{tabular}{|c|c|c|c|}
\hline\hspace{8mm}\textbf{\$2}\hspace{8mm} & \hspace{8mm}\textbf{\$3}\hspace{8mm} & \hspace{8mm}\textbf{\$4}\hspace{8mm} & \hspace{8mm}\textbf{\$5}\hspace{8mm} \\
\hline & & & \\ & & & \\
\hline & & & \\ & & & \\
\hline & & & \\ & & & \\
\hline & & & \\ & & & \\
\hline & & & \\ & & & \\
\hline & & & \\ & & & \\
\hline & & & \\ & & & \\
\hline

\end{tabular}
\end{center}

\item What does this program do?

\vspace{7mm}\hrule\vspace{7mm}\hrule\vspace{7mm}\hrule\vspace{7mm}\hrule
\vspace{7mm}\hrule\vspace{7mm}\hrule\vspace{3mm}

\end{enumerate}

\newpage
\item~
\begin{enumerate}
\item Indicate which lines of WRAMP assembler were generated for the given C code.

\begin{enumerate}
\item~
\vspace{7mm}\hrule\vspace{3mm}
\item~
\vspace{7mm}\hrule\vspace{3mm}
\item~
\vspace{7mm}\hrule\vspace{3mm}

\end{enumerate}

\item Draw a diagram to show the stack frame created by the function \texttt{max\_absolute}.

\begin{center}
 \psfig{figure=stack.eps, width=\linewidth}
\end{center}

\end{enumerate}

\newpage
\item~
\begin{enumerate}

\item What would be the new values of the registers after completing the described control steps.
\begin{center}
\begin{tabular}{|lp{1.5cm}|lp{1.5cm}|lp{1.5cm}|lp{1.5cm}|}
\hline
\$0 = & & \$1 = & & \$2 = & & \$3 = &  \\
& & & & & & & \\
& & & & & & & \\
\hline
\$4 = & & \$5 = & & \$6 = & & \$7 = & \\
& & & & & & & \\
& & & & & & & \\
\hline
\$8 = & & \$9 = & & \$10 = & & \$11 = & \\
& & & & & & & \\
& & & & & & & \\
\hline
\$12 = & & \$13 = & & \$sp = & & \$ra = & \\
& & & & & & & \\
& & & & & & & \\
\hline
PC = & & IR = & & TEMP = & &  & \\
& & & & & & & \\
& & & & & & & \\
\hline
\end{tabular}
\end{center}

\item Indicate the control steps necessary to load the next instruction into the Instruction Register, and increment the Program Counter.

\begin{center}
\begin{tabular}{|l|p{1.2cm}|p{1.2cm}|p{1.2cm}|p{1.2cm}|p{1.2cm}|}
\hline
\textbf{Signal} & \textbf{Step 1} & & & & \\
\hline
\texttt{a\_out} & & & & & \\
\hline
\texttt{sel\_a} & & & & & \\
\hline
\texttt{b\_out} & & & & & \\
\hline
\texttt{sel\_b} & & & & & \\
\hline
\texttt{c\_in} & & & & & \\
\hline
\texttt{sel\_c} & & & & & \\
\hline
\texttt{alu\_out} & & & & & \\
\hline
\texttt{alu\_func} & & & & & \\
\hline
\texttt{mem\_read} & & & & & \\
\hline
\texttt{mem\_write} & & & & & \\
\hline
\texttt{pc\_out} & & & & & \\
\hline
\texttt{pc\_in} & & & & & \\
\hline
\texttt{imm\_16\_out} & & & & & \\
\hline
\texttt{imm\_20\_out} & & & & & \\
\hline
\texttt{sign\_extend} & & & & & \\
\hline
\texttt{ir\_in} & & & & & \\
\hline
\texttt{temp\_out} & & & & & \\
\hline
\texttt{temp\_in} & & & & & \\
\hline
\end{tabular}
\end{center}

\newpage
\item Indicate the control steps necessary to execute the instruction \texttt{jal guesswhat}. Assume the instruction has been fetched,
and the Program Counter incremented.

\begin{center}
\begin{tabular}{|l|p{1.2cm}|p{1.2cm}|p{1.2cm}|p{1.2cm}|p{1.2cm}|}
\hline
\textbf{Signal} & \textbf{Step 1} & & & & \\
\hline
\texttt{a\_out} & & & & & \\
\hline
\texttt{sel\_a} & & & & & \\
\hline
\texttt{b\_out} & & & & & \\
\hline
\texttt{sel\_b} & & & & & \\
\hline
\texttt{c\_in} & & & & & \\
\hline
\texttt{sel\_c} & & & & & \\
\hline
\texttt{alu\_out} & & & & & \\
\hline
\texttt{alu\_func} & & & & & \\
\hline
\texttt{mem\_read} & & & & & \\
\hline
\texttt{mem\_write} & & & & & \\
\hline
\texttt{pc\_out} & & & & & \\
\hline
\texttt{pc\_in} & & & & & \\
\hline
\texttt{imm\_16\_out} & & & & & \\
\hline
\texttt{imm\_20\_out} & & & & & \\
\hline
\texttt{sign\_extend} & & & & & \\
\hline
\texttt{ir\_in} & & & & & \\
\hline
\texttt{temp\_out} & & & & & \\
\hline
\texttt{temp\_in} & & & & & \\
\hline
\end{tabular}
\end{center}

\item Indicate the control steps necessary to execute the instruction \texttt{lw \$4, 25(\$5)}. Assume the instruction has been fetched,
and the Program Counter incremented.

\begin{center}
\begin{tabular}{|l|p{1.2cm}|p{1.2cm}|p{1.2cm}|p{1.2cm}|p{1.2cm}|}
\hline
\textbf{Signal} & \textbf{Step 1} & & & & \\
\hline
\texttt{a\_out} & & & & & \\
\hline
\texttt{sel\_a} & & & & & \\
\hline
\texttt{b\_out} & & & & & \\
\hline
\texttt{sel\_b} & & & & & \\
\hline
\texttt{c\_in} & & & & & \\
\hline
\texttt{sel\_c} & & & & & \\
\hline
\texttt{alu\_out} & & & & & \\
\hline
\texttt{alu\_func} & & & & & \\
\hline
\texttt{mem\_read} & & & & & \\
\hline
\texttt{mem\_write} & & & & & \\
\hline
\texttt{pc\_out} & & & & & \\
\hline
\texttt{pc\_in} & & & & & \\
\hline
\texttt{imm\_16\_out} & & & & & \\
\hline
\texttt{imm\_20\_out} & & & & & \\
\hline
\texttt{sign\_extend} & & & & & \\
\hline
\texttt{ir\_in} & & & & & \\
\hline
\texttt{temp\_out} & & & & & \\
\hline
\texttt{temp\_in} & & & & & \\
\hline
\end{tabular}
\end{center}


\end{enumerate}

\end{enumerate}

\end{document}




