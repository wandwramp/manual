\documentclass[a4paper,10pt]{article}
\setlength{\parindent}{0ex}
\setlength{\parskip}{1.5ex}
\usepackage[dvips]{graphicx}
\usepackage{epsfig}
\usepackage{verbatim}
\usepackage[a4paper, hmargin=25mm, vmargin=30mm, nohead]{geometry}

\usepackage{fancyhdr}
\fancypagestyle{rcsfooters}{
\fancyfoot[L]{\small $ $RCSfile: data.tex,v $ $}
\fancyfoot[R]{\small $ $Revision: 1.1 $ $}
}

\renewcommand{\headrulewidth}{0pt}

%INCLUDE OUR GLOBALS
\usepackage{rex201}
\usepackage{styles201}
\usepackage{ex201}


\begin{document}

\EXHEADING{\DATANO}{\DATATITLE}{%
Verification Date:~\DATADUE\\
}

The objective of this assignment is to help you understand the 
different representations used to store
data within a computer system. This is to be achieved by getting
you to perform a set of number and
character conversions and inspect the characteristics of a 
number of files.
The tasks that you will be asked to perform are:

\begin{comment}
\begin{enumerate}
\item Convert a number from hexadecimal to binary
\item Convert a number from binary to hexadecimal
\item Convert a number from decimal to two's compliment
\item Convert a number from two's compliment to decimal
\item Convert a decimal value to an ASCII character
\item Convert an ASCII character to a decimal value
\item Convert a number from decimal to BCD
\item Convert a number from BCD to decimal
\item Convert a number from decimal to IEEE floating point (x 2)
\item  Convert a number from IEEE floating point to decimal (x 2)
\item  Identify the characteristics of an audio file
\item  Identify an object in a postscript file.
\end{enumerate}
\end{comment}

\begin{verbatim}
      o Convert a number from hexadecimal to binary
      o Convert a number from binary to hexadecimal
      o Convert a number from decimal to two's compliment
      o Convert a number from two's compliment to decimal
      o Convert a decimal value to an ASCII character
      o Convert an ASCII character to a decimal value
      o Convert a number from decimal to BCD
      o Convert a number from BCD to decimal
      o Convert a number from decimal to IEEE floating point (x 2)
      o Convert a number from IEEE floating point to decimal (x 2)
      o Identify the characteristics of an audio file
      o Identify an object in a postscript file.
\end{verbatim}

The actual assignment is an individualised test that 
you can obtain from the World Wide Web. The
test is a mastery style test. That is, to pass the 
test you must score 13/14 or better. However, there is
no limit to the number of times that you can sit the 
test. If you fail the test you will be given a
completely different version of the test next 
time you sit it. There is no time limit to sit a test (apart
from the final deadline of the \DATADUE) or 
any necessity to complete it in a single sitting. To obtain a
test you must to be logged into one of the 
machines in lab one, then go to the following web page:

\begin{verbatim}
     http://warlock.cs.waikato.ac.nz/~201/2003/data-rep.php
\end{verbatim}

You will then be asked your username. Once you have filled this out
and clicked on the "\src{Start Test}" button. You will then be given a
multi-choice test that you can either do by sitting at terminal or
printing it out so that you can do it at your own leisure. If you opt
for the second option, then when you think you have the correct
answers to all of the questions, you will need to re-sit the test by
returning to the above URL (you will be given the same questions). To
complete the test you will need to create your personal verification
code. This code allows the online test to verify that it has been
submitted by you. Follow the instructions on the page to create this
code. If you have any problems you should attend one of the in Lab
tutorials and the tutor will be able to help you get your code. Once
you have completed the test then click on the "Submit Test" button at
the bottom of the question sheet. You should not submit a test until
you are ready to have it marked as you will be given a completely
different test next time you try to re-sit it.  The test will then be
marked online and you will be given a mark plus feedback on which
questions you answered correctly and incorrectly. If you score 13 or
more out of 14 in a test then you will be given the verification for
this assignment. If you get 12 or less you should check out another
version of the test and try again. Note that there are at least 150
different versions of each of the 14 questions.  Also note that the
number of times that you have sat a new test is recorded.

\thispagestyle{rcsfooters}
\pagestyle{rcsfooters}
\end{document}
