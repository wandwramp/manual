\documentclass[12pt]{report}
\usepackage{a4}
\usepackage[dvips]{graphicx}
\usepackage[a4paper,hmargin=25mm,vmargin=20mm,nohead]{geometry}

% Define a subscript macro
\newcommand\subscr[1]{\raisebox{-0.5ex}{\small #1}}

% Define some macros to display register names ie. Rx with the x in subscript
\newcommand\regs{R\subscr{s}}
\newcommand\regd{R\subscr{d}}
\newcommand\regt{R\subscr{t}}


\begin{document}


\setcounter{secnumdepth}{0}
%\pagenumbering{roman}


\section{WRAMP Instruction Set Summary}


\subsection{Arithmetic Instructions}
\noindent
{\bf Addition}

\texttt{add \regd, \regs, \regt}\\
Put the sum of register \regs{} and register \regt{}
into register \regd{}.
\vspace{1ex}

%\noindent
%{\bf Addition, immediate}
\texttt{addi \regd, \regs, Immediate}\\
Put the sum of register \regs{} and the sign-extended immediate
into register \regd{}.
\vspace{1ex}

%\noindent
%{\bf Addition, unsigned} - 
\texttt{addu \regd, \regs, \regt}\\
Put the sum of register \regs{} and register \regt{}
into register \regd{}.
\vspace{1ex}

%\noindent
%{\bf Addition, unsigned, immediate} -
\texttt{addui \regd, \regs, Immediate}\\
Put the sum of register \regs{} and the zero-extended immediate
into register \regd{}.
\vspace{1ex}

\noindent
{\bf Subtraction}

\texttt{sub \regd, \regs, \regt}\\
Put the difference of register \regs{} and register \regt{}
into register \regd{}.
\vspace{1ex}

%\noindent
%{\bf Subtraction, immediate}
\texttt{subi \regd, \regs, Immediate}\\
Put the difference of register \regs{} and the sign-extended immediate
into register \regd{}.
\vspace{1ex}

%\noindent
%{\bf Subtraction, unsigned}
\texttt{subu \regd, \regs, \regt}\\
Put the difference of register \regs{} and register \regt{}
into register \regd{}.
\vspace{1ex}

%\noindent
%{\bf Subtraction, unsigned, immediate}
\texttt{subui \regd, \regs, Immediate}\\
Put the difference of register \regs{} and the zero-extended immediate
into register \regd{}.
\vspace{1ex}

\noindent
{\bf Multiplication}

\texttt{mult \regd, \regs, \regt}\\
Put the product of the signed multiplication of register \regs{} and register \regt{}
into register \regd{}.
\vspace{1ex}

%\noindent
%{\bf Multiplication, immediate}
\texttt{multi \regd, \regs, Immediate}\\
Put the product of the signed multiplication of register \regs{} and the sign-extended immediate
into register \regd{}.
\vspace{1ex}

%\noindent
%{\bf Multiplication, unsigned}
\texttt{multu \regd, \regs, \regt}\\
Put the product of the unsigned multiplication of register \regs{} and register \regt{}
into register \regd{}.
\vspace{1ex}

%\noindent
%{\bf Multiplication, unsigned, immediate}
\texttt{multui \regd, \regs, Immediate}\\
Put the product of the unsigned multiplication of register \regs{} and the zero-extended immediate
into register \regd{}.
\vspace{1ex}

\noindent
{\bf Division}

\texttt{div \regd, \regs, \regt}\\
Put the result of the signed integer division of register \regs{} by register \regt{}
into register \regd{}.
\vspace{1ex}

%\noindent
%{\bf Division, immediate}
\texttt{divi \regd, \regs, Immediate}\\
Put the result of the signed integer division of register \regs{} by the sign-extended immediate
into register \regd{}.
\vspace{1ex}

\newpage
%\noindent
%{\bf Division, unsigned}
\texttt{divu \regd, \regs, \regt}\\
Put the result of the unsigned division of register \regs{} by register \regt{}
into register \regd{}.
\vspace{1ex}

%\noindent
%{\bf Division, unsigned, immediate}
\texttt{divui \regd, \regs, Immediate}\\
Put the result of the unsigned division of register \regs{} by the zero-extended immediate
into register \regd{}.
\vspace{1ex}

\noindent
{\bf Remainder}

\texttt{rem \regd, \regs, \regt}\\
Put the remainder of the signed division of register \regs{} by register \regt{}
into register \regd{}.
\vspace{1ex}

%\noindent
%{\bf Remainder, immediate}
\texttt{remi \regd, \regs, Immediate}\\
Put the remainder of the signed division of register \regs{} by the sign-extended immediate
into register \regd{}.
\vspace{1ex}

%\noindent
%{\bf Remainder, unsigned}
\texttt{remu \regd, \regs, \regt}\\
Put the remainder of the unsigned division of register \regs{} by the register \regt{}
into register \regd{}.
\vspace{1ex}

%\noindent
%{\bf Remainder, unsigned, immediate}
\texttt{remui \regd, \regs, Immediate}\\
Put the remainder of the unsigned division of register \regs{} by the zero-extended immediate
into register \regd{}.
\vspace{1ex}

\noindent
{\bf Load high immediate}

\texttt{lhi \regd, Immediate}\\
Put the 16 bit immediate into the upper 16 bits of register \regd{},
and set the lower 16 bits to zero.
\vspace{1ex}

\noindent
{\bf Load address}

\texttt{la \regd, Address}\\
Put the zero-extended 20 bit address into register \regd{}.
\vspace{1ex}

\subsection{Bitwise instructions}

\noindent
{\bf And}

\texttt{and \regd, \regs, \regt}\\
Put the result of the logical AND of registers \regs{} and \regt{} into register \regd{}.
\vspace{1ex}

%\noindent
%{\bf And, immediate}
\texttt{andi \regd, \regs, Immediate}\\
Put the result of the logical AND of register \regs{} and the zero-extended immediate into register \regd{}.
\vspace{1ex}

\noindent
{\bf Or}

\texttt{or \regd, \regs, \regt}\\
Put the result of the logical OR of registers \regs{} and \regt{} into register \regd{}.
\vspace{1ex}

%\noindent
%{\bf Or, immediate}
\texttt{ori \regd, \regs, Immediate}\\
Put the result of the logical OR of register \regs{} and the zero-extended immediate into register \regd{}.
\vspace{1ex}

\noindent
{\bf Xor}

\texttt{xor \regd, \regs, \regt}\\
Put the result of the logical exclusive-OR of registers \regs{} and \regt{} into register \regd{}.
\vspace{1ex}

\newpage
%\noindent
%{\bf Xor, immediate}
\texttt{xori \regd, \regs, Immediate}\\
Put the result of the logical exclusive-OR of regisster \regs{} and the zero-extended immediate into register \regd{}.
\vspace{1ex}

\noindent
{\bf Shift left logical}

\texttt{sll \regd, \regs, \regt}\\
Shift the value in register \regs{} left by the unsigned value given by the
least significant 5 bits of register \regt{}, and put the result in register \regd{},
inserting zeros into the low order bits.
\vspace{1ex}

%\noindent
%{\bf Shift left logical, immediate} - 
\texttt{slli \regd, \regs, Immediate}\\
Shift the value in register \regs{} left by the unsigned value given by the
least significant 5 bits of the immediate, and put the result in register \regd{},
inserting zeros into the low order bits.
\vspace{1ex}

\noindent
{\bf Shift right logical}

\texttt{srl \regd, \regs, \regt}\\
Shift the value in register \regs{} right by the unsigned value given by the
least significant 5 bits of register \regt{}, and place the result in register \regd{},
inserting zeros into the high order bits.
\vspace{1ex}

%\noindent
%{\bf Shift right logical, immediate} - 
\texttt{srli \regd, \regs, Immediate}\\
Shift the value in register \regs{} right by the unsigned value given by the
least significant 5 bits of the immediate, and place the result in register \regd{},
inserting zeros into the high order bits.
\vspace{1ex}

\noindent
{\bf Shift right arithmetic}

\texttt{sra \regd, \regs, \regt}\\
Shift the value in register \regs{} right by the unsigned value given by the
least significant 5 bits of register \regt{}, and place the result in register \regd{},
sign-extending the high order bits.
\vspace{1ex}

%\noindent
%{\bf Shift right arithmetic, immediate} - 
\texttt{srai \regd, \regs, Immediate}\\
Shift the value in register \regs{} right by the unsigned value given by the
least significant 5 bits of the immediate, and place the result in register \regd{},
sign-extending the high order bits.
\vspace{1ex}

\subsection{Test instructions}

\noindent
{\bf Set on less than}

\texttt{slt \regd, \regs, \regt}\\
Set register \regd{} to 1 if register \regs{} is
less than register \regt{} according to a signed comparison, and 0 otherwise.
\vspace{1ex}

%\noindent
%{\bf Set on less than immediate} - 
\texttt{slti \regd, \regs, Immediate}\\
Set register \regd{} to 1 if register \regs{} is
less than the sign-extended immediate according to a signed comparison, and 0 otherwise.
\vspace{1ex}

%\noindent
%{\bf Set on less than, unsigned} -
\texttt{sltu \regd, \regs, \regt}\\
Set register \regd{} to 1 if register \regs{} is
less than register \regt{} according to an unsigned comparison, and 0 otherwise.
\vspace{1ex}

\newpage
%\noindent
%{\bf Set on less than, unsigned, immediate} - 
\texttt{sltui \regd, \regs, Immediate}\\
Set register \regd{} to 1 if register \regs{} is
less than the zero-extended immediate according to an unsigned comparison, and 0 otherwise.
\vspace{1ex}

\noindent
{\bf Set on greater than}

\texttt{sgt \regd, \regs, \regt}\\
Set register \regd{} to 1 if register \regs{} is
greater than register \regt{} according to a signed comparison, and 0 otherwise.
\vspace{1ex}

%\noindent
%{\bf Set on greater than, immediate} - 
\texttt{sgti \regd, \regs, Immediate}\\
Set register \regd{} to 1 if register \regs{} is
greater than the sign-extended immediate according to a signed comparison, and 0 otherwise.
\vspace{1ex}

%\noindent
%{\bf Set on greater than, unsigned} - 
\texttt{sgtu \regd, \regs, \regt}\\
Set register \regd{} to 1 if register \regs{} is
greater than register \regt{} according to an unsigned comparison, and 0 otherwise.
\vspace{1ex}

%\noindent
%{\bf Set on greater than, unsigned, immediate} - 
\texttt{sgtui \regd, \regs, Immediate}\\
Set register \regd{} to 1 if register \regs{} is
greater than the zero-extended immediate according to an unsigned comparison, and 0 otherwise.
\vspace{1ex}

\noindent
{\bf Set on less than or equal to}

\texttt{sle \regd, \regs, \regt}\\
Set register \regd{} to 1 if register \regs{} is
less than or equal to register \regt{} according to a signed comparison, and 0 otherwise.
\vspace{1ex}

%\noindent
%{\bf Set on less than or equal to, immediate} - 
\texttt{slei \regd, \regs, Immediate}\\
Set register \regd{} to 1 if register \regs{} is
less than or equal to the sign-extended immediate according to a signed comparison, and 0 otherwise.
\vspace{1ex}

%\noindent
%{\bf Set on less than or equal to, unsigned} - 
\texttt{sleu \regd, \regs, \regt}\\
Set register \regd{} to 1 if register \regs{} is
less than or equal to register \regt{} according to an unsigned comparison, and 0 otherwise.
\vspace{1ex}

%\noindent
%{\bf Set on less than or equal to, unsigned, immediate} - 
\texttt{sleui \regd, \regs, Immediate}\\
Set register \regd{} to 1 if register \regs{} is
less than or equal to the zero-extended immediate according to an unsigned comparison, and 0 otherwise.
\vspace{1ex}

\noindent
{\bf Set on greater than or equal to}

\texttt{sge \regd, \regs, \regt}\\
Set register \regd{} to 1 if register \regs{} is
greater than or equal to register \regt{} according to a signed comparison, and 0 otherwise.
\vspace{1ex}

%\noindent
%{\bf Set on greater than or equal to immediate} - 
\texttt{sgei \regd, \regs, Immediate}\\
Set register \regd{} to 1 if register \regs{} is
greater than or equal to the sign-extended immediate according to a signed comparison, and 0 otherwise.
\vspace{1ex}

%\noindent
%{\bf Set on greater than or equal to, unsigned} - 
\texttt{sgeu \regd, \regs, \regt}\\
Set register \regd{} to 1 if register \regs{} is
greater than or equal to register \regt{} according to an unsigned comparison, and 0 otherwise.
\vspace{1ex}

%\noindent
%{\bf Set on greater than or equal to, unsigned, immediate} - 
\texttt{sgeui \regd, \regs, Immediate}\\
Set register \regd{} to 1 if register \regs{} is
greater than or equal to the zero-extended immediate according to an unsigned comparison, and 0 otherwise.
\vspace{1ex}

\newpage
\noindent
{\bf Set on equal to}

\texttt{seq \regd, \regs, \regt}\\
Set register \regd{} to 1 if register \regs{} is
equal to register \regt{} according to a signed comparison, and 0 otherwise.
\vspace{1ex}

%\noindent
%{\bf Set on equal to immediate} -
\texttt{seqi \regd, \regs, Immediate}\\
Set register \regd{} to 1 if register \regs{} is
equal to the sign-extended immediate according to a signed comparison, and 0 otherwise.
\vspace{1ex}

%\noindent
%{\bf Set on equal to, unsigned} - 
\texttt{sequ \regd, \regs, \regt}\\
Set register \regd{} to 1 if register \regs{} is
equal to register \regt{} according to an unsigned comparison, and 0 otherwise.
\vspace{1ex}

%\noindent
%{\bf Set on equal to, unsigned, immediate} - 
\texttt{sequi \regd, \regs, Immediate}\\
Set register \regd{} to 1 if register \regs{} is
equal to the zero-extended immediate according to an unsigned comparison, and 0 otherwise.
\vspace{1ex}

\noindent
{\bf Set on not equal to}

\texttt{sne \regd, \regs, \regt}\\
Set register \regd{} to 1 if register \regs{} is
not equal to register \regt{} according to a signed comparison, and 0 otherwise.
\vspace{1ex}

%\noindent
%{\bf Set on not equal to immediate} - 
\texttt{snei \regd, \regs, Immediate}\\
Set register \regd{} to 1 if register \regs{} is
not equal to the sign-extended immediate according to a signed comparison, and 0 otherwise.
\vspace{1ex}

%\noindent
%{\bf Set on not equal to, unsigned} - 
\texttt{sneu \regd, \regs, \regt}\\
Set register \regd{} to 1 if register \regs{} is
not equal to register \regt{} according to an unsigned comparison, and 0 otherwise.
\vspace{1ex}

%\noindent
%{\bf Set on not equal to, unsigned, immediate} - 
\texttt{sneui \regd, \regs, Immediate}\\
Set register \regd{} to 1 if register \regs{} is
not equal to the zero-extended immediate according to an unsigned comparison, and 0 otherwise.
\vspace{1ex}

\subsection{Branch instructions}

\noindent
{\bf Jump}

\texttt{j Address}\\
Unconditionally jump to the instruction whose address is given by the address field.
\vspace{1ex}

%\noindent
%{\bf Jump to register} - 
\texttt{jr \regs}\\
Unconditionally jump to the instruction whose
address is in the least significant 20 bits of register \regs{}.
\vspace{1ex}

\noindent
{\bf Jump and link}

\texttt{jal Address}\\
Unconditionally jump to the instruction whose address is given by the address field.
Save the address of the next instruction in register \texttt{\$ra}.
\vspace{1ex}

%\noindent
%{\bf Jump and link register} - 
\texttt{jalr \regs}\\
Unconditionally jump to the instruction whose
address is in the least significant 20 bits of register \regs{}.
Save the address of the next instruction in register \texttt{\$ra}.
\vspace{1ex}

\newpage
\noindent
{\bf Branch on equal to zero}

\texttt{beqz \regs, Offset}\\
Conditionally branch the number of instructions specified by the
sign-extended offset if register \regs{} is equal to 0.
\vspace{1ex}

\noindent
{\bf Branch on not equal to zero}

\texttt{bnez \regs, Offset}\\
Conditionally branch the number of instructions specified by the
sign-extended offset if register \regs{} is not equal to 0.
\vspace{1ex}

\subsection{Memory instructions}

\noindent
{\bf Load word}

\texttt{lw \regd, Offset(\regs{})}\\
Add the contents of register \regs{} and the sign-extended offset to
give an effective address. Load the contents of the memory location
specified by this effective address into register \regd{}.
\vspace{1ex}

\noindent
{\bf Store word}

\texttt{sw \regd, Offset(\regs{})}\\
Add the contents of register \regs{} and the sign-extended offset to
give an effective address. Store the contents of register \regd{} into
the memory location specified by this effective address.
\vspace{1ex}

\subsection{Special instructions}

\noindent
{\bf Move general register to special register}

\texttt{movgs \regd, \regs}\\
Copy the contents of general purpose register \regs{} into special purpose register \regd{}.
\vspace{1ex}

\noindent
{\bf Move special register to general register}

\texttt{movsg \regd, \regs}\\
Copy the contents of special purpose register \regs{} into general purpose register \regd{}.
\vspace{1ex}

\noindent
{\bf Break}

\texttt{break}\\
Generate a breakpoint exception, transferring control to the exception handler.
\vspace{1ex}

\noindent
{\bf System call}

\texttt{syscall}\\
Generate a system call exception, transferring control to the exception handler.
\vspace{1ex}

\noindent
{\bf Return from exception}

\texttt{rfe}\\
Restore the saved interrupt enable and kernel/user mode bits and jump to
the instruction at the address specified in the special register \texttt{\$ear}.
\vspace{1ex}

\end{document}
